\chapter{Evaluation and Analysis}
\label{chap:evaluation}

After having implemented the prototype, we evaluated it by taking into account our initial design and use cases. In the next two sections we discuss, how well the prototype already fits to our use cases and how the design guidelines from chapter \ref{chap:generaldesign} were implemented. 


%%% ------------- SECTION -------------
\section{Use Case Analysis}

In this section we will discuss what use cases from chapter \ref{usecase} the prototype has implemented and what is still missing. The more, we will discuss how the user can access the described functionality as well as some benefits and drawbacks of the implementation. For the remainder of this section we describe the current and active user as $A$ and any other user $A$ deals with as $B$.

When $A$ opens our website the login screen will appear. There $A$ can enter its user name and password to log into the system. This view also allows to register a new user via a click on the register button. Once logged in, there is a logout button in the action box on the top right of every page of the website. Hence use cases $1.1$, $1.2$ and $1.3$ for registration, login and logout are implemented as required. 
\newline
Use case $1.4$ is described further down. To view the health record as use case $1.5$ demands we introduced the concept of spaces, as described in chapter \ref{chap:prototype}. Once $A$ clicks on the spaces link in the navigation the \emph{all} space appears, where all the latest entries are shown in a list view similar to a Twitter feed. \newline
$1.6$ and $1.7$ ask for editing and deletion of entries. This functionality has only been implemented on the server side for the prototype yet. There is no GUI component which uses this functionality. The user can only edit the circle and space assignments for the individual entries. The reason for this is that there are still too many open questions which need to be resolved with the stakeholders. Editing becomes a problem if the data is not generated by the users themselves. It would be rather critical if the user could change the findings of its doctor, a drug prescription or data from an application (e.g. calories burned while jogging). Deleting is a problem on all social networks nowadays. Some of them just do not show the item to the user anymore but do not actually delete the entry for years. For our platform we would of course prefer, if entries would be kept anonymous for research aspects. These questions have to be resolved by lawyers before we can implement the respective use cases on the system.

Changing profile data as use case $1.8$ specifies is done in the profile page of the web site. There the user can change profile details, the user icon and the password. The more the user can share its data with any circles.\newline
Search as described in $1.9$ is done in two ways. If the user wants to search for other users, e.g. to assign them to a circle, this can be done in the circle page of the web site. There is a search form on top of that page, which will present a list of matching users with their first and last name (or company name if it is an institute) and the user icon. Searching for applications and entries is done via the general search in the action box on the top right of every page. $1.10$ is presented further down.

Use cases $1.11$ up to $1.16$ describe access categories and actions on them. As described in chapter \ref{chap:prototype} in the prototype we call access categories circles. Most of the action is done on the specific circle page of the client. Here, the user can see all its circles (at least the four predefined ones), click on them to see the assigned users or create new circles by clicking on the \emph{+} circle. With a click on a circle the users assigned to it are displayed and the user can edit circle information such as the name, description and colour in which they are displayed. For predefined circles only the colour can be changed. Also in this menu user created circles can be deleted. To add a new user to a circle $A$ can either search for user $B$ or click on any other circle to see its members. The assignment to the new circle is then done via drag-and-drop. The user picks the rectangle, where the user details is presented, and drags it over to the desired circle. By dropping it on the circle, the selected user will be assigned to it. Removing a user $B$ from a circle works similarly. $A$ clicks on the desired circle, drags the rectangle of $B$ to the right and drops it on the \emph{remove from circle} field which will pop up. \newline
The entries to circle assignment is done in the spaces view. In every space the user will see the feed of all the entries within that particular space. Next to each entry there is a button to change the circle assignment. A click on this button shows all the different circles with a check box next to them. The entries will be shared with all the selected circles. Another way of defining circle assignments is via the application detail page. $A$ can select circles for applications. This has the result that the system will assign the selected circles to every entry this particular application will create for $A$.

Case $1.17$ describes the folders, which are called spaces in the prototype. Dealing with spaces is done on the spaces page accessible via the navigation on the left. First the user will see the so called \emph{all} space where all the entries are listed. On top is a tab view to switch between the different spaces. Next to it is a gear icon that allows to change which spaces are shown in the tab list and in what order. The more the menu behind this button allows creating new spaces. In order to assign different entries to these spaces, the user can click on the spaces button next to each entry and select the desired ones. This is the same procedure as for the circle assignment. Also the user can select spaces for applications, so all new entries by this application will automatically be assigned to the desired space. The space view has a feature for filtering the entries in the feed. This filter option allows to see data of other users they shared with $A$. This feature is the implementation of requirement $1.10$.

$1.18$ and $1.19$ deal with installing and removing applications from $A$`s profile. In order to install an application $A$ can go into the market place on the web site and select the desired application. In the detail view $A$ will see a description, update information, the number of times this application has been installed so far and other information about the application. Also in this view there is an install button. By a click on it, the application will be installed on $A$`s account. To uninstall the application again, $A$ can follow the same procedure. In the \emph{My Apps} page we list all the applications $A$ has installed to facilitate searching for them. For the actual task of uninstalling we show an uninstall button in the detail view of an application. All the entries generated by the application will remain in $A$`s profile after the application was uninstalled. \newline
Since applications are the main module to add new record entries to the health record of a user, we can now discuss requirement $1.4$. If $A$ wants to add any new entry, the first task is to install an application. Then depending on the application $A$ can manually enter data via a form or similar. If the application uses B2B communication $A$ has to link its Healthbank profile with the external provider. If this was successful the application will automatically add new entries to the user`s record.

Sending messages among users as desired by use case $1.20$ is done via our messaging system. This can be accessed by clicking the inbox link in the navigation or the message icon in the action box. The messaging system is kept very simple. Users can send new messages by selecting a recipient, setting a header and message and clicking the send button. The inbox shows all the received messages in a list. By clicking on one of these messages in the inbox, the detail message view appears, where the user can read the entire message and reply to it. 

Deletion of the entire profile as required in $1.21$ is not yet supported by the prototype. The reason is a similar one as the reason for not allowing deletion of entries yet. We need clarification on what is happening with the user`s entries, when the profile is deleted. Again this is a question for the lawyers. 

Institutes can add new entries to a user via an application we implemented especially for institute users to meet use case $2.1$. The application is called \emph{Patient Record} and provides a simple form to enter the necessary data. Institute users will find the application in the market place and are able to install it from there. In order to save a record entry to another user, this user has to have the institute assigned to one of its circles. The reason is simply that we want the user to have full control on its data and who is able to add data to its account. In a normal patient to doctor relation this is usually no problem, since the patient will probably share health data with the doctor.\newline
Profiles of institutes are very similar to normal users. The difference is mainly that an institute has a company name rather than first and last name and that all kind of personal data specific to an individual (such as height, weight, insurance, etc.) are omitted in favour of a company description. Since use case $2.2$ does not specify any details in which the two user types shall be different, we did not make matters more complicated than needed. \newline
The query engine, $2.3$ is asking for, was implemented in a rather basic way. Institute users have the possibility to search for other users by setting constraints on age, height, etc. attributes and by defining keywords, the user must have in at least one record. Additionally they can query for record entries by providing relevant keywords or using the text search framework from MongoDB. All the data returned is taken from users who actively allowed researchers to be able to inspect their data. 

Adding, editing and removing applications by institutes as required in use cases $2.4$ to $2.6$ is implemented in the application section of the web site. When an institute is logged in they will see the link to add a new application in the navigation. There they can fill out a form according to the definition of an application in chapter \ref{chap:prototype}. With the same form institutes can create new visualizations. The distinction is done by selecting the right option in a dropdown menu. The page called \emph{uploaded} shows the institutes all the application they have added to the Healthbank system yet. From this view they can select an application to edit by clicking on the gear icon next to the entry in the list. The editing form is very similar to the new application form but contains an additional field. This field allows to tell the users what features that changed in comparison to the last version. Deletion of an application or visualization as required in use case $2.6$ is only implemented on the server side of the prototype so far. From the GUI it is not possible yet because of similar reasons as discussed before. We need additional discussion with the stakeholders to define, what happens with the entries the application created and if we are legally required to remove all traces of the files uploaded by the authors. 

Use case $2.7$ discusses adding new entries to the users profile by an application. We implemented two ways of doing so as mentioned earlier. Either the user can enter data manually in the application context (e.g. the provided application to add medical records) or there is a B2B communication between the third-party server and our server. The latter case is supported by the generation of one time access tokens as described in chapter \ref{chap:prototype}. Use case $2.8$ is supported by our concept of visualizations which can be integrated in an iframe on any of the spaces pages (apart from \emph{all}). Visualizations get all the record entries in the current space via the client API and can illustrate all or parts of them. 


%%% ------------- SECTION -------------
\section{Prototype Evaluation}

\subsection{Architecture and Technologies}

For the prototype we used MongoDB which allows to save both structured and unstructured data in a JSON-style document-oriented storage. This turned out to be a very good fit for the system. With the Java Driver provided by MongoDB it was straight forward to write even rather difficult queries. We did not have to spend a lot of time on the schema and table design as we would have in a SQL database. Adding and dropping fields during the implementation process as well as inspecting and manipulating single or multiple entries in the database turned out to be very well manageable with the help of the MongoDB Shell application. Another advantage over SQL databases is the underlying JSON data structure of MongoDB which fits perfectly well to the JSON communication structure we used for the communication between server and client.\newline
We used Java Servlets to implement the server running on a Tomcat server. This allowed us to implement our REST API with HTTP GET and POST requests as requested by our guidelines in chapter \ref{chap:generaldesign}. With the help of Eclipse debugging and testing was possible without a lot of extra work thanks to the smooth integration of Tomcat in Eclipse.\newline
Having the client completely decoupled from the server led to a couple of problems but allowed us to test the parts individually. The possibilities by HTML 5 in combination with CSS 3 and JavaScript frameworks such as jQuery are enormous and allow to create dynamic, responsive and state of the art web sites. With the help of the Chrome Developer Tools debugging on the client worked quite well even though JavaScript debugging and especially testing is still quite a mess. 

In general we can confidently state that the designed architecture worked out very well for the prototype and the different concepts could be integrated with little problems on top of it. Though it would be wise to give some more thoughts on the architecture of the user`s management. With our prototype we only use one user collection to store patients, health providers, researchers, application developers and other kind of users. A doctor`s office or a hospital employs multiple people, they have information such as opening hours, different expertise or special medical equipment. Furthermore, they are not interested in adding new record entries to their profile and have different functionalities then regular users. Hence it might be advisable to separate the way users and institutes are stored in the database. 

From the technological point of view there is also one thing we suggest to spend some more time to think about. On the server side of the prototype we implemented many things on our own and had to deal with parsing both URL parameters and database entries. This was sometimes rather cumbersome. That is why for a next prototype or a first product we advise to use a slightly different approach. We suggest to use frameworks such as Java Spring~\cite{spring}, Apache Lucene together with Solr~\cite{lucene} or Elastic Search~\cite{elasticsearch} right from the beginning to implement the REST API. Even though these frameworks come with a lot of baggage they provide useful features which make everyday life easier.

\subsection{Design and Usability}

The focus on the prototype was clearly to create a proof of concept rather than to make the website look nice and shiny. Hence we did not spend a lot of time on design questions. Nevertheless usability was an important factor and wherever possible we applied navigation and interaction principles the users are already familiar with from other web sites. For the circles page e.g. we used a similar look and feel as Google+ and we introduced drag-and-drop to give the user some interaction possibilities. The navigation was intended to be always visible and at a prominent spot on the left. This way users always know how to get from one page to another. With the action box on the top right of the screen we wanted to give the user full control over their profile, messages and a universal search box right at hand on every page to reduce the amount of clicks needed. We used CSS 3 animations, transitions and other components when possible to improve the user experience and to make the page more dynamic. The system was mainly developed and optimized using the browser Google Chrome but we did also test it on Mozilla Firefox. 

To help new users to get started we added tooltips and help messages. The first time a user visits any of the main pages a greeting message with many tips will be shown which explains how to use the particular page. This includes installing new applications, adding friends to a circle, creating new spaces and adding a visualization to it, filtering record entries and displaying entries of other users or sending messages.



%%% ------------- SECTION -------------
\section{Problems during Implementation}

A design with decoupled components, as we decided to have for the prototype, is very handy if one decides to create e.g. another GUI (mobile version, Android/iOS etc.). On the other hand it led to certain problems such as, above all, the same-origin policy~\cite{sameOrigin} defined by the W3C (World Wide Web Consortium). This policy was developed to protect users by disallowing cross-origin requests on a website. In other words, a website loaded from server $A$ is not allowed to load data from server $B$ if $A$ and $B$ have different scheme, host or port. Or as the W3C states it: 
\begin{quote}
"The prohibition on receiving information is intended to prevent malicious web sites from reading confidential information from other web sites, 
but also prevents web content from legitimately reading information offered by other web sites."
\end{quote}
In the end we kept the client still separated from the server but running on the same Tomcat server as the REST API. This way we had the same host and port and could prevent the same-origin policy problems which occurred especially in the case of file uploads. 

Another issue was to come up with a working design for different screen sizes. We tried to work with percent attributes in the CSS as much as possible but still had a couple of calls by the first users who complained that a certain component is not shown as supposed too. For time reasons we only tested the system properly on Windows 7 running Google Chrome and Firefox on a 14.1" and 24" screen as well as on a 7" Android tablet running Chrome. 


%%% Local Variables:
%%% mode: latex
%%% TeX-master: "thesis"
%%% End: