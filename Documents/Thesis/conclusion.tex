	
\chapter{Conclusion and Future Work} 
\label{chap:conclusion}
 
\section{Conclusion}
In this thesis we have introduced the concept of the Healthbank platform. Healthbank tries to overcome the problem of storing health data for individual people. We have introduced a general architecture for such a platform and compared it to existing projects such as Indivo, Microsoft HealthVault or Google Health. The main contribution by the proposed system is the strong focus on the users. The users decide who is allowed to add data to their record and who is allowed to see their data. Thereby we accept both structured and unstructured data from all kind of third-party application providers. As a proof of concept, we implemented a first prototype on top of the proposed architecture. For this prototype we built a REST API using MongoDB and Java Servlets for the backend and a website using HTML 5 and jQuery for the client. With the developed system users can register themselves, log in to and out of the system, create record entries using external applications, organize and illustrate them in spaces and share them via the circles concept.  The prototype showed that most of the use cases are feasible. Nevertheless for some use cases more discussion is needed concerning legal and social aspects. With its ability to add applications and visualizations by third-party providers such as e.g. Runtastic, there are numerous possibilities to extend the system. If the users accept the platform and provide their health information, it will serve as a big database for all kind of symptoms and diseases which could help researches to find new ways of curing people. Organising this huge amount of structured and unstructured data and being able to do analysis on top of it will be one of the most difficult problem yet to solve in an efficient way.\newline


\section{Future Work}

The system we implemented for this thesis is a prototype and still missing a few very important functionalities. The Healthbank system is designed to deal with a huge amount of users since every person in Switzerland and globally theoretically has a health record and hence is a potential customer. For the system this means that one has to deal with concurrency and replication to guarantee availability and a short response time. Luckily MongoDB provides a rather well replication and a clever sharding system which allows to run on several machines in parallel. But if there are thousands of requests for the page per minute, the system has to cope with it and be robust. Load distribution and caching are among others methods that should be taken into account for the final going live. 
\newline
Another important fact that one should take more seriously in future versions is security. Login and session management is a very important aspect of a big system with such important data. For the prototype we set minimal security standards for the log in procedure such as never sending passwords in clear text and dealing with a session key. Nevertheless there is much more behind this process. One should discuss systems like OAuth or use a commercial product. Besides log in and session one has to lay a close focus on security for entries, applications, file upload and sharing. Furthermore HTTPS connections will be a must have for the final product.
\newline
The prototype server deals a lot with GET and POST request and uses variable URL parameters for the different requests. To implement the REST concept the way it is intended to one should e.g. make profile information accessible via a call to $\ldots/user/username$ rather than providing the user identification as a URL parameter to a user API call. The same goes for applications and entries of the health record. In addition the PUT and DELETE requests should be used more frequently for a better understanding of the API.
\newline
Once the system is live and the database fills up with a lot of data it is essential, that a clever search engine is built on top of it. This will distinctively improve the overall search and user experience. The prototype already supports a basic search facility taking into account the most important keys of the different applications. However, as soon as the system grows and has much more data, this will probably get to its limits rather quickly. To improve the search engine it is important to work closely with application providers as well as with the users to optimize for user friendliness and motivate other providers to write applications for us. As mentioned in the prototype evaluation we suggest to use search platforms such as Apache Solr or Elastic Search on top of the MongoDB to improve query performance and to offer more search options.
\newline
There is more discussion needed with the stakeholders and lawyers to introduce the missing functionality for some of the use cases we defined. Though, there are many more features which would work well with the Healthbank system but we did not have enough time to implement. Hence we end this document by providing a list of ideas for features which could be built on top of the system:
\begin{itemize}
	\item A rating system and reviews for applications and visualizations.
	\item User feedback (rating and reviews) for health providers such as doctors and hospitals.
	\item A search engine extension the implements a location based services to find local institutes that match the need of a user. 
	\item An implementation of the health information platform described in the overall architecture\ref{HIP} containing e.g. a wiki for diseases, a drug register, the latest research results, etc. 
	\item A rich desktop interface for hospitals and doctor`s offices that help their administrative staff to easily connect to our system and upload the patient`s data.
	\item Integration of the eHealth and ePatientendossier applications in Switzerland. These applications contain information saved on the insurance cards of individual people. We could use it to provide access to the user`s profile in case of emergencies.
	\item Improve the organ donation system. Integration of the Swisstransplant (and similar in other countries) would possibly help to find new donors and ease up the process, since all necessary information is stored in a single database on our system.
	\item A collaboration with the health insurances would provide them with the data they need and allow the users to find the perfect fit for them. A system like Comparis could analyse the user`s data and suggest the best insurance company.
	\item An integration of e.g. the Rega, the paraplegic centre in Nottwil or any other health care providers would help them to find more members and complete the user profiles with important data. 
	\item The possibility for institutes to synchronize or at least publish their calendars on the system would help users to find an appointment and the institutes to have a tighter schedule.
	\item Integration of pharmacies together with saving prescriptions by doctors in the system would allow users to have everything online and accessible all the time. 
\end{itemize}

\newpage
\section{Acknowledgments}
I would like to thank my supervisor Prof. Dr. Donald Kossmann from the ETH Systems Group. He always had time for a short meeting and helped and motivated me throughout my master studies at ETH.\newline
A special thank goes to Nahas Nasri and Catherine Zwahlen from the Healthbank cooperative for their support and ideas.\newline