\begin{abstract}

Today everything gets digital, we share our thoughts and pictures, we book flights, we do shopping, we do our payments and many other things, all online. But something is stilling missing in our shiny online world, health data. Or do you have control over your latest x-ray, blood test, ECG or drug prescription? Do you know exactly where this data is stored and who has access to it?

Ever since the World Wide Web emerged and got successful there were different attempts by private companies and governments to organize health data and to standardize it. Nevertheless there is no global or even national system yet, where a user can have access to all of its very personal health data, especially here in Europe. The reasons for that include, among others, that users tend to have trust issues with profit-oriented companies such as Google. In addition governments did focus too much on saving money and optimizing for insurance companies rather than building a platform for the user. 

A patient`s health record is still distributed over many different health care providers such as the family doctors, hospitals, dentists, pharmacies or wellness providers. For this reason doctors and hospitals lose a lot of time to get the patient`s health history. Missing information does therefore not only cost them a lot of money but might also cause long-term damage or even lead to death if the patient gets treated in the wrong way.

With this thesis we introduce the idea of the Healthbank system. We try to overcome the stated problems and allow the users to gather all their health data in a secure, centralized way. Thereby the data shall actually be owned, be accessible and controllable by the users themselves. The system shall pile up all kind of structured as well as unstructured data of patients and make it visible for the user via multiple client platforms. These platforms shall allow to install external applications to aggregate other kind of health data such as e.g. the user`s jogging record. Furthermore, with this platform the users shall be able to share their data with other users or health care providers such as doctors or hospitals in a fine-grained way. At the same time the control over the data shall always entirely rely on the user.

Apart from describing the idea of the Healthbank system, in this thesis we compare it to other approaches, define the overall structure of such a system and present in detail a first Healthbank prototype.

\end{abstract}
